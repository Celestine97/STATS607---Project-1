\documentclass[main]{subfiles}

\begin{document}
\subsection{Motivation: fMRI data}
\label{sec-data}

We examine fMRI data from Kay et al (2008).  In their experiments, a subject is presented with an image, and
fMRI signals in the primary visual cortex (V1) are measured. Our focus will be restricted to voxels 1-20, and we wish to predict
the responses of these twenty voxels to each image. Features from each image were extracted using a Gabor wavelet pyramid,
resulting in $p = 10921$ features for each image, and there were a total of $n = 1705$ images.

Neurons in this region have a localized receptive field,
and cortical columns in V1 have been well modeled using Gabor transformations. Hence it is
reasonable to assume the fMRI response in a certain voxel is (roughly) a linear function of a small
number of Gabor transforms of the image.

Therefore, we chose to fit a LASSO model to predict voxel responses for this particular application.
Furthermore, we investigate whether or not choosing a pre-factor $K$ can improve
our LASSO predictions. In section \ref{sec-single_repsonse} we fit independent LASSOs to each voxel, and investigate the benefits of
additionally choosing $K$.

However, we also have knowledge that adjacent voxels are likely to be strongly correlated, so instead of fitting 20 independent
prediction models, we would like to take advantage of these correlations to improve predictive accuracy.
We explore ways to extend the curds-and-whey
method proposed by Breiman \& Friedman (1997) to this high-dimensional, sparse case.

\end{document}
